\section{Einleitung}
Ziel des Projekts war es, einen \glqq{}Fuchsjagdpeilsender\grqq{} zu konzipieren und aufzubauen.
Es handelt sich dabei um einen Sender, der bei Amateurfunker-Wettbewerben -- in der Regel in einem
Wald -- versteckt und von den Teilnehmern mit Hilfe von geeigneten Empfängern gesucht wird.\\
Die Frequenz des gesendeten Signals sollte sich in dem für Amateurfunker freigegebenen
80-Meter-Band befinden und die Sendeleistung mindestens 1W betragen. Um für den Wettbewerb
tauglich zu sein, muss der Fuchsjagdpeilsender in regelmäßigen Abständen ein Morsezeichen senden.
Es soll möglich sein, die Intervalllänge und das zu sendende Symbol am Peilsender einzustellen.
Außerdem muss es möglich sein, den Sender mit einer handelsüblichen 9V-Batterie über einen 
ausreichenden Zeitraum zu betreiben und er darf nicht zu schwer oder unhandlich sein, sodass eine
Person mehrere davon transportieren kann. \\\\
Das Projekt ist Teil der Veranstaltung \glqq{}Elektronikentwicklung - eine praktische 
Einführung\grqq{} und soll den Studenten die Möglichkeit geben, praxisnähere Erfahrungen zu sammeln,
als sonst im Studium üblich.